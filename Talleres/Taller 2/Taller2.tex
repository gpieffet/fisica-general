\documentclass[10pt]{article}
%\documentclass[prd,amssymb,preprint]{revtex4}
\setlength{\topmargin}{-0.4in}
\setlength{\oddsidemargin}{0.15in} 
\setlength{\textwidth}{5.7in}
\setlength{\textheight}{9.1in}
\usepackage{epsfig}
\usepackage{hyperref}
\usepackage{multicol} 
\usepackage[spanish]{babel}
\usepackage[utf8]{inputenc}
\usepackage{nicefrac}

\pagestyle{empty}		% Remove page number

\begin{document}

\noindent
{\Large \bf Taller de Física General}
\vskip 0.3cm
\noindent
{\large \bf Taller \# 2}
\vskip 0.5cm
\noindent
%{\bf } \\
{\it Temas: Vectores.}\\
{\it Profesores:  G. Pieffet}\\


% {\bf 1ra Parte.} Repaso de vectores.
\begin{enumerate}
% \bf 1ra Parte. Repaso de vectores.

% Pregunta 1
\item >Si $|\vec B|= 18$ m y $\vec B$ apunta en la dirección del eje $x$ negativo, que valen $B_x$ y $B_y$?

% Pregunta 2
\item Encuentre la magnitud y la dirección (ángulo formado con el eje {\em x} positivo) de los siguientes vectores:\label{itm:vector_def}
\begin{multicols}{2} 
\begin{enumerate}
	\item $\vec a = (2,2)$
	\item $\vec b = (-1, \sqrt 2)$
	\item $\vec c = (-1, \nicefrac{1}{2})$
	\item $\vec d = (0, 2)$
\end{enumerate}
\end{multicols}

\item Exprese en componentes cartesianas los siguientes vectores:\\
$\vec A = 2$ unidades, en dirección $\alpha = 60^{\circ}$\\
$\vec B = 8$ unidades, en dirección $\beta = 230^{\circ}$


% Pregunta 4
\item Con los vectores del ejercicio \ref{itm:vector_def}, haga las siguientes operaciones y calcule la magnitud y dirección del vector resultante de cada operación:
\begin{multicols}{2}
\begin{enumerate}
	\item $\vec a + 2\vec b - \vec c$
	\item $-2\vec a + 3 \vec b - \vec c$
	% \item $\vec a + \vec b - \vec c - \vec d$
	% \item $-3\vec b + \vec c + 2\vec d$
\end{enumerate}
\end{multicols}

% Pregunta 5
\item Si viaja una distancia $r = 22$ km en linea recta desde el origen en una dirección $35^\circ$ al Sur del Oeste, >cual es su posición en coordenadas cartesianas (en $x$ e $y$)? 

\item Exprese en componentes cartesianas los siguientes vectores $\vec a$, $\vec b$, $\vec c$ y $\vec d$ con magnitudes $a = 5$, $b = 4$, $c = 3$ y $d = \sqrt{20}$
\begin{figure}[!ht]
	\centering\includegraphics[width=0.24\textwidth]{./fig/vector_suma1}
\end{figure}


% Pregunta 6
\item Calcule el vector resultante (las componentes, la magnitud y la dirección) de la sumatoria de los vectores $\vec A$, $\vec F$ y $\vec M$ representados en la figura \ref{fig:vector}.\label{itm:vector_grafica}
\begin{figure}[!ht]
	\centering\includegraphics[width=0.24\textwidth]{./fig/vector_suma3-2}
	\caption{Vectores $\vec A$, $\vec F$ y $\vec M$ del ejercicio \ref{itm:vector_grafica}.}\label{fig:vector}
\end{figure}


\end{enumerate}
\end{document}



