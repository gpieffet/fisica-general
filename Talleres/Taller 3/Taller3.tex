\documentclass[10pt]{article}
%\documentclass[prd,amssymb,preprint]{revtex4}
\setlength{\topmargin}{-0.4in}
\setlength{\oddsidemargin}{0.15in} 
\setlength{\textwidth}{5.7in}
\setlength{\textheight}{9.1in}
\usepackage{epsfig}
\usepackage{hyperref}
\usepackage{multicol} 
\usepackage[spanish]{babel}
\usepackage[utf8]{inputenc}
\usepackage{nicefrac}

\pagestyle{empty}		% Remove page number

\begin{document}

\noindent
{\Large \bf Taller de Física General}
\vskip 0.3cm
\noindent
{\large \bf Taller \# 3}
\vskip 0.5cm
\noindent
%{\bf } \\
{\it Temas: Gráficas, cálculos de errores, ajuste lineal.}\\
{\it Profesores:  G. Pieffet}\\


\begin{enumerate}

\item Escribir las medidas siguientes de forma apropiada:
\begin{enumerate}
	\item $A = 123.89456 \pm 0.054\;\rm{cm}^2$
	\item $t = 247.871 \pm 0.21$ s
	\item $m = 7548.21 \pm 26.2$ kg
\end{enumerate}
	
\item Si se mide el diámetro de un circulo y se obtiene $d = 5.0 \pm 0.1$ cm, cuál es el perímetro del circulo así que el error asociado?  

\item En un experimento para medir el calor latente del hielo (también llamado entalpía de fusión y que corresponde a la energía para pasar 1 kg de hielo de fase solida a fase liquida), un estudiante pone un bloque de hielo en un vaso de icopor con agua y observa el cambio de temperatura mientras que el hielo se derrite. Para determinar la masa del bloque de hielo, el estudiante mide la masa del vaso de agua antes y después de poner el bloque de hielo. Las dos medidas son:\\
masa (vaso icopor + agua)$ = m_1 = 203 \pm 2$ g\\
masa (vaso icopor + agua + hielo)$ = m_2 = 246 \pm 3$ g\\
Cuál es la respuesta del estudiante para la masa del hielo, así que el incertidumbre?

\item Convertir los errores absolutos en las medidas de la velocidad de 2 carros en una pista en errores relativos y errores relativos porcentuales:
\begin{enumerate}
	\item $v_1 = 55 \pm 2$ cm/s
	\item $v_2 = 20 \pm 2$ cm/s
	\item Si la energía cinética medida es igual a $E_c = 4.58\;\rm{J} \pm 2\%$, escribe esta medida en términos del error absoluto.
\end{enumerate}

\item Si se miden una masa $m = 7.5 \pm 0.3$ kg y la aceleración gravitacional $g = 9.81 \pm 0.02\;\rm{ m/s^2}$, cuál es el peso $P = mg$ correspondiente y el error asociado?  

\item \textbf{Regresión lineal o ajuste lineal:} método de mínimos cuadrados.\\
Considere el siguiente conjunto de datos para el movimiento de un vehículo.
\begin{center}
\begin{tabular}{| l | r | r | r | r | r | r |}
	\hline
	t(s) & $1.2$ & $2.6$ & $7.5$ & $8.2$ & $10.8$ & $15.4$ \\ \hline
	x(m) & $4$   & $8.1$ & $15.5$& $18.4$& $27.9$ & $39.8$ \\
	\hline
\end{tabular}
\end{center}
\begin{enumerate}
	\item Coloque los puntos en una gráfica.
	\item Encuentre la ecuación de la recta $y = mx + b$ que mejor ajusta los datos experimentales utilizando el método de mínimos cuadrados. La pendiente se calcula por la ecuación:
	\[ m = \frac{n\sum xy - \sum x \sum y}{n\sum x^2 - (\sum x)^2} \]
	mientras que el punto de corte con el eje $y$ se puede calcular con:
	\[ b = \bar y - m \bar x \]
\end{enumerate}

\end{enumerate}
\end{document}



