\documentclass[10pt]{article}
%\documentclass[prd,amssymb,preprint]{revtex4}
\setlength{\topmargin}{-0.4in}
\setlength{\oddsidemargin}{0.15in} 
\setlength{\textwidth}{5.7in}
\setlength{\textheight}{9.1in}
\usepackage{epsfig}
\usepackage{hyperref}
\usepackage{multicol} 
\usepackage[spanish]{babel}
\usepackage[utf8]{inputenc}
\usepackage{nicefrac}

\pagestyle{empty}		% Remove page number

\begin{document}

\noindent
{\Large \bf Taller de Física General}
\vskip 0.3cm
\noindent
{\large \bf Taller \# 5}
\vskip 0.5cm
\noindent
%{\bf } \\
{\it Temas: Cinemática 2-D.}\\
{\it Profesores:  G. Pieffet}\\


% \begin{enumerate}
\noindent
\begin{minipage}{0.6\linewidth}
	\begin{enumerate}
		\item Hallar a qué velocidad hay que realizar un tiro parabólico para que llegue a una altura máxima de $100$ m si el ángulo de tiro es de $30^{\circ}$
		\item Hallar a que ángulo hay que realizar un tiro parabólico para que el \emph{alcance} y la \emph{altura} máxima sean iguales.
	\end{enumerate}
\end{minipage}
\begin{minipage}{0.35\linewidth}
	\hspace{10mm}\includegraphics[width=1\textwidth]{./fig/parabolico1}
\end{minipage}

% Nueva pregunta
\begin{enumerate}
\item [3.] Un futbolista patea un balón que cae a una distancia de $35$ m y $15$ s después. Con que ángulo y cuál fue la velocidad inicial con la que salió el balón.

% Nueva pregunta
\item [4.] Una pelota se lanza horizontalmente desde la azotea de un edificio de $35$ metros de altura. La pelota golpea el suelo en un punto a $80$ metros de la base del edificio. Encuentre: 
\begin{enumerate}
	\item El tiempo que la pelota permanece en vuelo (el tiempo de vuelo). 
	\item Su velocidad inicial.
	\item Las componentes $v_x$ y $v_y$ de la velocidad justo antes de que la pelota pegue en el suelo así que la velocidad final correspondiente.
\end{enumerate}
\end{enumerate}

% Nueva pregunta
\noindent
\begin{minipage}{0.6\linewidth}
	\begin{enumerate}
		\item [5.] Jimmy está ubicado $27.85$ m de la parte inferior de la colina, mientras que Billy se encuentra $11.14$ m metros arriba de la misma. Jimmy está en el origen de un sistema de coordenadas $x$,$y$. Si Jimmy lanza una manzana a Billy con un ángulo de $50^{\circ}$ con respecto a la horizontal, con que velocidad debe lanzar la manzana para que pueda llegar a Billy?\\
\emph{Nota: Billy NO tiene que estar ubicado en el punto mas alto de la trayectoria.}
	\end{enumerate}
\end{minipage}
\begin{minipage}{0.35\linewidth}
	\hspace{10mm} \includegraphics[size=0.5]{./fig/parabolico3}
\end{minipage}

% Nueva pregunta
% \begin{minipage}{0.35\linewidth}
% 	\centering \includegraphics[width=1\textwidth]{./fig/parabolico2}
% \end{minipage}
% \begin{minipage}{0.6\linewidth}
% 	\begin{enumerate}
% 		\item [4.] Un arquero quiere efectuar un tiro parabólico entre dos acantilados tal y como indica la figura. El acantilado de la izquierda se halla $4$ m por arriba con respecto al de la derecha. Si el arquero solo puede disparar con un ángulo de $30^{\circ}$ y quiere lanzar las flechas a $5$ m del acantilado de la derecha, calcula con que velocidad mínima ha de lanzarlas. Calcula el tiempo de vuelo.
% 	\end{enumerate}
% \end{minipage}

\end{document}



