\documentclass[10pt]{article}
%\documentclass[prd,amssymb,preprint]{revtex4}
\setlength{\topmargin}{-0.4in}
\setlength{\oddsidemargin}{0.15in} 
\setlength{\textwidth}{5.7in}
\setlength{\textheight}{9.1in}
\usepackage{epsfig}
\usepackage{hyperref}
\usepackage{multicol} 
\usepackage[spanish]{babel}
\usepackage[utf8]{inputenc}
\usepackage{nicefrac}

\pagestyle{empty}		% Remove page number

\begin{document}

\noindent
{\Large \bf Taller de Física General}
\vskip 0.3cm
\noindent
{\large \bf Taller \# 4}
\vskip 0.5cm
\noindent
%{\bf } \\
{\it Temas: Cinemática 1D, distancia, velocidad, aceleración.}\\
{\it Profesores:  G. Pieffet}\\


\begin{enumerate}
\item Draw a horizontal x-axis. An object is placed at the origin. It is then pushed $2$ cm to the right, then $4$ cm to the left, then $5$ cm to the right. Mark the position of the object after each of these three displacements (choose a convenient scalew on the axis).
\begin{enumerate}
	\item What is the displacement vector (magnitud and sign) of each of those three pushes?
	\item What is the (total) resultant displacement vector (magnitud and direction)?
	\item What is the total distance covered in the three pushes?
\end{enumerate}


\item A car covers the distance of $290$ km in three hours. Find the average speed of the car in km per hour and in meter per second.


\item A car travels $0.2$ m in $0.01$ s. What is its instantaneous velocity?


\item Two trains leave the train station at the same time, moving in opposite directions. One moves at $30$ m/s, and the other at $40$ m/s. Write the equation of motion for each train, using the same x-axis.


\item A car maker claims that his car can accelerate from rest to $90$ kilometers per hour in $6.2$ seconds. What is the acceleration of the car? What is the distance that the car travels in those $6.2$ seconds?


\item A car travels at $45$ m/s. The driver then accelerates the car at $2.3$ m/s$^2$ along $150$ m. What is the final velocity of the car?


\item An airplane must reach a velocity of $75$ m/s in order to take off. What is the shortest runway that it can use if it accelerates at a constant acceleration of $12$ m/s$^2$?

\item Si un cuerpo se deja caer libremente, calcule su posición y velocidad después de $1$, $2$, $3$ y $4$ segundos. Tome el punto de partida como origen, el eje $y$ vertical y la dirección positiva hacia arriba (en este caso $\vec a = \vec g$ y $g = 10$ m/s).


\item Un aeroplano necesita $600$ m para despegar del campo.
\begin{enumerate}
	\item > Cuál es su aceleración (suponiendo que es constante) si necesita $15$ s para alcanzar una velocidad suficiente para despegar?
	\item >Cuál es la velocidad?
\end{enumerate}


\item An elevator started its way up from rest at the ground floor, moving in constant acceleration. When it reached the height of $2$ m, its velocity was $1.5$ m/s. Find:
\begin{enumerate}
	\item Its acceleration.
	\item The time it took to climb the $2$ m.
\end{enumerate}


\item En este problema tenemos un carro que se mueve de maneras diferentes en un eje horizontal, mostrado en cada figura. Ubicamos el eje de coordenadas de manera que el eje positivo de las \emph{x} y apunta a la derecha. Para cada situación se pueden dar seis formas diferentes de representar la información que lo describe: Un diagrama de posiciones, una gráfica de velocidad versus tiempo, un conjunto de vectores para la velocidad instantánea, una gráfica de aceleración versus tiempo, una descripción en palabras y un par de flechas que representan las direcciones de la velocidad y la aceleración.
El primer ejercicio es un ejemplo con toda la información completa. Entienda las 6 representaciones. En cada uno de los casos subsecuentes complete la información sobre las seis formas de representación sobre el movimiento.
\begin{figure}[!ht]
	\centering\includegraphics[width=0.6\textwidth]{./fig/6repr_ejemplo}
	\caption {Ejemplo completo.}
\end{figure}
\begin{figure}[!ht]
	\centering \includegraphics[width=0.6\textwidth]{./fig/6repr_prob1}
	% \includegraphics[width=0.5\textwidth]{./fig/6repr_prob2}
	\caption {Problema 1.}
\end{figure}
\begin{figure}[!ht]
	\centering\includegraphics[width=0.6\textwidth]{./fig/6repr_prob2}
	\caption {Problema 2.}
\end{figure}


\end{enumerate}
\end{document}



