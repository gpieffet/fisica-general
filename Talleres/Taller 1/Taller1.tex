\documentclass[10pt, twocolumn]{article}
%\documentclass[prd,amssymb,preprint]{revtex4}
\setlength{\topmargin}{-0.5in}
\setlength{\oddsidemargin}{-0.4in} 
\setlength{\textwidth}{7.3in}
\setlength{\textheight}{9.0in}
\usepackage{graphicx}
\usepackage{hyperref}
\usepackage{multicol}
\usepackage{subcaption}
\usepackage[spanish]{babel}
\usepackage[utf8]{inputenc}
\usepackage{nicefrac}

\pagestyle{empty}		% Remove page number

\begin{document}

\noindent
{\Large \bf Taller de Física General}
\vskip 0.3cm
\noindent
{\large \bf Taller \# 1}
\vskip 0.5cm
\noindent
%{\bf } \\
{\it Temas: Geometría, Trigonometría y Algebra.}\\
{\it Profesores:  Gilles Pieffet}\\


{\bf 1ra Parte.} Geometría.
\begin{enumerate}

% Pregunta 1
\item Calcule la longitud de una rueda de $90$ cm de diámetro.

% Pregunta 2
\item > Cuales pares de ángulos son verticales?
\begin{figure}[h!]
	\centering\includegraphics[width=0.27\textwidth]{./fig/vertical_angles}
\end{figure}

%Pregunta 3
\item Se tiene que echar cemento para construir un anden de $1.5$ m de largo alrededor de una piscina rectangular de $10\;\rm{m} \times 25\;\rm{m}$. El anden tiene un grosor de $20$ cm. > Cuantos m$^3$ de cemento se necesitan?

%Pregunta 4
\item Encuentre el valor del ángulo $\alpha$ en cada una de las figuras.
\begin{figure}[!ht]
	\centering
	\includegraphics[width=0.45\textwidth]{./fig/alpha-1}
\end{figure}
\begin{figure}[!ht]
	\centering
	\includegraphics[width=0.3\textwidth]{./fig/alpha-2}
\end{figure}
\begin{figure}[!ht]
	\centering
	\includegraphics[width=0.3\textwidth]{./fig/alpha-3}
\end{figure}
\begin{figure}[!ht]
	\centering
	\includegraphics[width=0.3\textwidth]{./fig/alpha-4}
\end{figure}

\newpage

\noindent {\bf 2da Parte.} Conversión de unidades.
% Pregunta 9
\item Convierte: 
\begin{enumerate}
	\item 1 año en segundos
	\item 1 segundo en años
	\item 5 metros en centímetros
	\item 2 decímetros en metros
	\item $15$ ms en s
	\item $20$ cm en m
	\item $0.5$ m en mm
	\item $0.5\ \rm mm^2$ en $\rm dm^2$
	\item $70$ cm/s en m/ms
\end{enumerate}

\noindent {\bf 3ra Parte.} Trigonometría.
\item Uno de los ángulos en un triángulo recto es $36^{\circ}$. Utilizando una calculadora, encuentre la razón entre:
\begin{enumerate} 
	\item el cateto opuesto a este ángulo y la hipotenusa
	\item el cateto adyacente y la hipotenusa
	\item el cateto opuesto y el cateto adyacente al ángulo
\end{enumerate}

\item La hipotenusa en un triángulo recto es $7.5$ m, y el cateto adyacente a uno ángulo es $4.5$ m.
\begin{enumerate} 
	\item Encuentre la longitud del otro cateto.
	\item Encuentre el seno del ángulo entre la hipotenusa y el cateto de $4.5$ m.
\end{enumerate}

\newpage

\noindent {\bf 4ta Parte.} Algebra.
\item Expresar relaciones con formulas.
\begin{enumerate} 
	\item > Cuantos días $d$ hay en $s$ semanas?
	\item > Cuantos horas $h$ hay en $d$ días?
	\item > A cuantos días $d$ corresponden $s$ semanas y $m$ meses?
	\item > Cuantas patas $p$ tienen juntos $v$ vacas, $c$ cerdos y $g$ gatos?
\end{enumerate}

\item Resuelva las siguientes ecuaciones a una incógnita:
\begin{multicols}{2}
\begin{enumerate}
	\item $46 - x = 0$
	\item $x + 321 = 543$
	\item $71 = x + 24$
	\item $4x = 150$
	\item $\displaystyle \frac x5 = 9$
	\item $\displaystyle x\cdot\frac32 = 6$
	\item $\displaystyle \frac 5x = 9$
	\item $\displaystyle \frac 95 = \frac 3x$
	\item $\displaystyle 2 \cdot \frac 38 x = 11$
	\item $4x +9 = 12 x$
	\item $\displaystyle \frac{x+3}{2} = 15$
	\item $\displaystyle \frac{2x -9}{6} = \frac{4x + 2}{3}$
	\item $\displaystyle \frac{2x -9}{6} = \frac{4x + 2}{3} + 1$
	\item $\displaystyle  4(2x + 8) = 7$
	\item $\displaystyle \frac {2}{x + 3} = 15$
	\item $\displaystyle \frac 12 + \frac 1x = \frac 14$
\end{enumerate}
\end{multicols}

\item Manipulando Formulas
\begin{enumerate}
	\item Dado $\displaystyle S = v\cdot t$, despejar $t$. Despejar $v$.
	\item Dado $\displaystyle S = \frac 12 at^2$, despejar $a$. Despejar $t$.
	\item Dado $\displaystyle S = v_0 + \frac 12 at^2$, despejar $a$. Despejar $v_0$.
	\item Dado $\displaystyle \frac{P_1V_1}{T_1} = \frac{P_2V_2}{T_2}$, despejar $V_1$. Despejar $T_2$.
\end{enumerate}

\item Resolver los siguientes sistemas de ecuaciones con 2 incognitas.
\begin{multicols}{2}
\begin{enumerate}
	\item 	$x + y = 9$\\
			$x = 3$
	\item 	$2x + 3y = 17$\\
			$y = 2$
	\item 	$x - y = 9$\\
			$x - 2 = 9 - x$
	\item 	$7x + 2y = 15$\\
			$2x - 1 = 3$
	\item 	$2x + 3y = 9$\\
			$y = x$
	\item 	$4x + 6y = 18$\\
			$x - y +3 = 3$
	\item 	$x +y = 4$\\
			$x - y = 0$
	\item 	$2x - 3y = -1$\\
			$5x + 5y = 10$
\end{enumerate}
\end{multicols}

\item Entendiendo formulas.\\
En las formulas siguientes puede averiguar la relación entre las variables, aunque no conoce la significación de estas variables.
\begin{enumerate}
	\item El voltaje $V$ entre los terminales de un resistor se puede calcular por la formula $V = I\cdot R$, donde $I$ es la corriente a través del resistor y $R$ es la resistancia del resistor.\\
	\em{Verdadero} o \em{Falso?}\\
	El voltaje es proporcional a la corriente. [V / F]\\
	El voltaje es inversamente proporcional a la corriente. [V / F]
	\item La distancia $d$ recorrida durante un de tiempo $t$ por un carro acelerando desde el reposo es dado por la ecuación $d = 0.5\cdot a\cdot t^2$, donde $a$ es la aceleración del carro. Si se triplica el tiempo, cual sería la distancia $d'$ recorrido durante este nuevo tiempo? Cual sería la diferencia entre las dos distancias recorridas? Cual sería cambio relativo entre las dos distancias recorridas?
\end{enumerate}

\end{enumerate}
\end{document}

%%%%%%%%%%%%%%%%%%%%%%%%%%%%%%%%%%%%%%%%%%%%%%%%%%%%%%%%%%%%%%%%
%%%%%%%%%%%%%%%%%%%%%%%%%%%%%%%%%%%%%%%%%%%%%%%%%%%%%%%%%%%%%%%%
%%%%%%%%%%%%%%%%%%%%%%%%%%%%%%%%%%%%%%%%%%%%%%%%%%%%%%%%%%%%%%%%

\noindent {\bf 3ra Parte.} Conversión de unidades.
% Pregunta 10
\item Convierte: 
\begin{enumerate}
	\item $15\ \mu$s en s
	\item $10$ nC en mC
	\item $0.5\ \rm mm^2$ en $\rm dm^2$
	\item $70$ cm/s en m/ms
\end{enumerate}
	





