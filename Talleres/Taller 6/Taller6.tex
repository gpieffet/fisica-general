\documentclass[10pt]{article}
%\documentclass[prd,amssymb,preprint]{revtex4}
\setlength{\topmargin}{-0.4in}
\setlength{\oddsidemargin}{0.15in} 
\setlength{\textwidth}{5.7in}
\setlength{\textheight}{9.1in}
\usepackage{epsfig}
\usepackage{hyperref}
\usepackage{multicol} 
\usepackage[spanish]{babel}
\usepackage[utf8]{inputenc}
\usepackage{nicefrac}

\pagestyle{empty}		% Remove page number

\begin{document}

\noindent
{\Large \bf Taller de Física General}
\vskip 0.3cm
\noindent
{\large \bf Taller \# 6}
\vskip 0.5cm
\noindent
%{\bf } \\
{\it Temas: Leyes de Newton, dinámica.}\\
{\it Profesores:  G. Pieffet}\\


\begin{enumerate}
\item Calcule la aceleración que produce una fuerza de $5$ N a un cuerpo cuya masa es de $1000$~g. Expresar el resultado en m/s$^2$.

\item Calcule la masa de un cuerpo si al recibir una fuerza de $200$ N le produce una aceleración de $300$ cm/s$^2$. Exprese el resultado en kg.

\item Un automóvil de $1250$ kg se mueve con rapidez inicial de $80$ km/h > Qué fuerza de frenado se requiere para detener el automóvil tras haber recorrido $20$ m?
	
\item Un ascensor de $800$ kg es levantado verticalmente mediante un cable. Encuentre la aceleración e indique la dirección de movimiento del ascensor si la tensión en el cable es: a) $9000$ N, b) $7840$ N y c) $2000$ N.

\item Una gran esfera para demolición de $4090$ kg está sujeta por dos cables de acero como se muestra en la figura. Calcule la tensión de cada uno de los cables.
\begin{figure}[!ht]
	\centering
	\includegraphics[width=0.40\textwidth]{./fig/grua}
\end{figure}

% \item En la figura se muestra el diseño de una propuesta para la entrada de un museo de historia donde se ha querido colocar una escultura en bronce que tiene una masa de $2500$~kg sobre una placa rectangular de $800$ kg de concreto. Estos a su vez reposan sobre las dos columnas inclinadas un ángulo de $70^{\circ}$ con la horizontal. > Cuál es la fuerza que debe soportar cada una de las columnas?
% \begin{figure}[!ht]
% 	\centering
% 	\includegraphics[width=0.55\textwidth]{./fig/estatua}
% \end{figure}

\end{enumerate}
\end{document}



